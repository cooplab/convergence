\documentclass[a4paper,10pt]{article}
\usepackage[T1]{fontenc}
\usepackage[utf8]{inputenc}
\usepackage{amsmath,mathrsfs,bm}
\usepackage[dvipsnames]{xcolor}
\usepackage{cleveref}
\usepackage{graphicx}
\usepackage{fullpage}
%\usepackage{natbib}
\usepackage{float}
\usepackage{subcaption}
\usepackage{multirow} 
\usepackage{hyperref}
\usepackage{bibunits}
\usepackage{csvsimple}
\usepackage[superscript,biblabel]{cite}

\newcommand{\jb}[1]{{\color{blue} (#1)} }
\newcommand{\nec}[1]{{\color{purple} (#1)} }
\newcommand{\gc}[1]{{\color{red} #1}}
\newcommand{\crefrangeconjunction}{--}
\begin{document}

\title{Notes on Information Theory in Evolution }


\date{}


\maketitle

Kullback–Leibler divergence 
\begin{equation}
\mathscr{D}(q^{\prime} \|  q) = \sum_i q_i \log \left( \frac{q_i^{\prime}}{q_i} \right)  \label{eqn:KL}
\end{equation}
\section{Frank 2012}
So understanding of Frank is that he makes the point that we can
express the gain in information by considering 

\begin{align}
\Delta_S \bar{z} &= \sum_i q_i z_i \label{eqn:Delta_z}\\
&= Cov(w,z) / \bar{w} \\
&=\beta_{zw} V_w /\bar{w}
\end{align}
where middle line is Robertson-Price identity.\\

He choose to consider $z$ to be the log fitness, he doesnt note this is
Malthusian fitness (but I think it is).  We can write the log-fitness
as 
\begin{equation}
w_i = \bar{w}\frac{q_i^{\prime}}{q_i}
\end{equation} 
found by rearranging $q_i^{\prime} = q_i \big(w_i/\bar{w}\big)$. So that log
fitness can be written as 
\begin{equation}
m_i = \log(w_i) = \log(\bar{w}) + log(q_i^{\prime}/q_i)
\end{equation}
So the change in mean log-fitness ($z$), by eqn \eqref{eqn:Delta_z}, is
\begin{align}
\Delta_S \bar{m} &= \sum_i \Delta q_i \log \left( \frac{q_i^{\prime}}{q_i}
\right) \\
&\sum_i  (q_i^{\prime} - q_i) \log \left( \frac{q_i^{\prime}}{q_i} \right)\\
\end{align}
by rearranging 
\begin{align}
\Delta_S \bar{m} &= \sum_i  q_i^{\prime} \log \left(
  \frac{q_i^{\prime}}{q_i} \right) +
  \sum_i  q_i \log \left( \frac{q_i}{q_i^{\prime}} \right)\\
& = \mathscr{D}(q^{\prime} \|  q)  + \mathscr{D}(q \|  q^{\prime})
\end{align}
from our definition of KL-divergence (eqn \eqref{eqn:KL}). This is a
symmetric measure of the information difference, the Jefferys
divergence, $J(q^{\prime},~q)$. Therefore,
\begin{equation}
J(q^{\prime},~q) = \Delta_S \bar{m} =  Cov(w,\log(w)) / \bar{w} 
\end{equation}

%%Might be worth checking out if we can get further,
%% https://stats.stackexchange.com/questions/137002/covariance-of-natural-logarithms-how-to-estimate-sigma-mathbf-x-mathbf-y
%%https://stats.stackexchange.com/questions/46874/covariance-of-transformed-random-variables/46930#46930
%% Also there{\prime}s a bunch of related facts about variance of log
%% transformed variables.

\end{document}
