
Evolution (journal) has a commentary section, seems like something like that could work.

Why are we impressed by convergent evolution? Well it's because we get
to see selection act repeatedly to shape new mutations and standing
variation into adaptations in similar ways. This is particularly
impressive when we see convergence to a particular environment. Convergence helps us build
evidence that the phenotype is an adaptation, and form evidence that
adaptation is `for' increasing survival/fitness in particular
environment. 

How much do we say about genetic drift as a source of apparent convergence (and/or release from constraint).

Two distinct, but related questions, can we say that a phenotype/variant has been under selection. How many distinct instances of selection have there been. 
Related Q: as the more independent instances, the more evidence we have that it's
selected being selected.  

When can we say that selection has been convergent in among populations?
Can we formalize this Q by thinking about the overlap in selective
deaths/births underlying the adaptative evolution of a particular trait, allele, or set of alleles within a population?

Why is this question easier when we have a resolved phylogeny with
no-incomplete lineage sorting.
Well in that case the selective births and deaths are clearly
independent. E.g. the individuals who lived/died, due to differential
predation pressure, to drive the adaptation of light coloured fur in
arctic foxes and hares were clearly different sets of individuals
(being foxes and hares respectively). 

%Useful bib of convergence
%http://www.oxfordbibliographies.com/view/document/obo-9780199941728/obo-9780199941728-0038.xml

If we have a tree of drift (w. no gene flow), seeing non-sisters sharing an selective
event can be ``enough''. E.g. if both populations show a sweep, not
shared with sisters, then we have evidence that selection has occurred
in both pops independently. 

\paragraph{What do we mean, or can we say about, convergence when we only see the selected allele}

\paragraph{What do we mean, or can we say about, convergence when we see the linked variation/sweep?}

\paragraph{What do we mean by convergence when we are thinking of quantitative traits?}
Should we also cover when phenotypes alone are seen? in that case using covar matrix can help in QST style analysis.
Idea of double sign test?



\paragraph{Issues about meaning of convergence when there is incomplete lineage sorting} 
%http://www.indiana.edu/~hahnlab/Publications/HahnNakhleh2016.pdf


\paragraph{Long term signal?}
Can long term selection/drift on trait or variant, erode signal of shared
history. E.g. id stabilizing selection acts separately on a shared
trait in two (now independent populations) can we fix alternate
solutions, even though we original shared pool. Similar Q about shared
haplotypes whittled away by recom/migration. 
--One Q is do we care? 