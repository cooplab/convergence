
Evolution (journal) has a commentary section, seems like something like that could work.



%Useful bib of convergence
%http://www.oxfordbibliographies.com/view/document/obo-9780199941728/obo-9780199941728-0038.xml

How much do we say about genetic drift as a source of apparent convergence (and/or release from constraint).

two distinct, but related questions, can we say that a phenotype/variant has been under selection. How many distinct instances of selection have there been. 
Related Q: as the more independent instances, the more evidence we have that it's
selected being selected.  

When can we say that selection has been convergent in among populations?
Why is this question easier when we have a resolved phylogeny with
no-incomplete lineage sorting.
Can we formalize this Q by thinking about the overlap in selective deaths/births underlying the emergence of a particular allele or set of alleles within a population?


\paragraph{What do we mean, or can we say about, convergence when we only see the selected allele}

\paragraph{What do we mean, or can we say about, convergence when we see the linked variation/sweep?}

\paragraph{What do we mean by convergence when we are thinking of quantitative traits?}
Should we also cover when phenotypes alone are seen? in that case using covar matrix can help in QST style analysis.
Idea of double sign test?

Can long term selection/drift on trait or variant, erode signal of shared
history. E.g. id stabilizing selection acts separately on a shared
trait in two (now independent populations) can we fix alternate
solutions, even though we original shared pool. Similar Q about shared
haplotypes whittled away by recom

\paragraph{Issues about meaning of convergence when there is incomplete lineage sorting} 
%http://www.indiana.edu/~hahnlab/Publications/HahnNakhleh2016.pdf

